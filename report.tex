\documentclass[10pt,a4paper]{article}

\usepackage[hidelinks]{hyperref}
\usepackage{amsmath}
\usepackage[margin=1.5in]{geometry}


\begin{document}
\twocolumn
\title{Mobile Robot Systems Mini Project 5}
\author{Sam Sully (sjs252), Paul Durbaba (pd452), Luke Dunsmore (ldd25)}
\date{Lent 2020}
\maketitle
\section{Introduction}
\section{Localisation}
This section of the project was developed by Sam Sully (sjs252). The approach used was a combination of the sensor-based particle filter used in exercise 1 and the range and bearing approach presented in Dr Prorok's thesis~\cite{prorok}.

The particle filter works by randomly picking samples (particles) from a proposal distribution and then computing the probability that each particle is correct based on measurements from the robot's sensors. We then re-sample the particles, replacing the less likely ones with more likely ones.

Our approach used a known map, the range sensors measurements from the robot's LIDAR and the range and bearing measurements between robots. The weight of the robot is first calculated using the range sensor measurements using a Gaussian measurement model. The weight is given by the below formula:
\[
	w_i = \sum_{s_{j} \in \mathrm{Sensors}}\Phi(R(i,j), s_{ij}, \sigma^2)
\]
where $w_i$ is the weight of particle i, $s_{ij}$ is the distance recorded by sensor j on the robot, $\Phi(x,\mu,\sigma)$ is the Gaussian PDF with mean $\mu$ and standard deviation $\sigma$ and $R(i,j)$ is the ray traced distance from particle $i$ in the direction of sensor $j$.

We then refine the weight if there are any other robots in range and line-of-sight (these robots are referred to as the neighbours, the set $N_i$ represents robot $i$'s neighbours.). To do this we compute a second weight, $\bar{w_i}$ using the below formula:
\[
	\bar{w_i} = \sum_{r_j \in N_i}\sum_{p_k \in r_j}\Phi\left(
	\begin{bmatrix}
		D_i(p_k)\\
		\Theta_i(p_k)
	\end{bmatrix},
	\begin{bmatrix}
		d_j\\
		\theta_j
	\end{bmatrix},
	\xi
	\right)
\]
where $p_k$ ranges over the set of particles from robot $r_j$, $d_j$ is the received distance between this robot and robot $r_j$, $\theta_j$ is the received bearing of this robot from $r_j$, $D_i(p_k)$ is the distance between the particle $i$ on this robot and the particle $p_k$ from the other robot, $\Theta_i(p_k)$ is the bearing between the particle $i$ and the particle $p_k$ on the other robot and $\xi$ is the covariance matrix. I have omitted normalisation factors.

We then take these two quantities and multiply them together to get the final weight.
\section{Centralised Navigation}
\section{Decentralised Navigation}
\begin{thebibliography}{9}
\bibitem{prorok} A. Prorok, Models and Algorithms for Ultra-Wideband Localization in Single- and Multi-Robot Systems. 2013.
\end{thebibliography}
\end{document}